\documentclass[10pt,twoside,slovak,a4paper]{article}

\usepackage[slovak]{babel}
\usepackage[IL2]{fontenc}
\usepackage[utf8]{inputenc}
\usepackage{graphicx}
\usepackage{url}
\usepackage{hyperref}

\usepackage{cite}

\pagestyle{headings}

\title{Moderné vzdelávanie pomocou online kurzov\thanks{Semestrálny projekt v predmete Metódy inžinierskej práce, ak. rok 2020/21, vedenie: Michal Hatala}}

\author{Adam Michalák\\[2pt]
	{\small Slovenská technická univerzita v Bratislave}\\
	{\small Fakulta informatiky a informačných technológií}\\
	{\small \texttt{xmichalaka@stuba.sk}}
	}

\date{\small 16. december 2020}



\begin{document}

\maketitle

\begin{abstract}
Online kurzy sa v posledných rokoch stali veľkou časťou moderného vzdelávania. Je to hlavne kvôli ich dostupnosti, na internete sú tisíce stránok s rôznymi kurzami zo všetkých možných odvetví. Každý človek s počítačom a pripojením na internet k ním má prístup. Je to veľmi užitočné hlavne vo svete IT, kde sú nové technológie vyvíjané každý deň. Svet IT sa denno-denne posúva ďalej a preto je dôležité mať prístup k informáciám z pohodlia domova. Tento článok sa zameria hlavne na to, prečo by mal každý využívať túto možnosť vzdelávania.
\end{abstract}


\section{Úvod}
Pozerajúc sa na najnovšie trendy za posledné desaťročie, dištančné vzdelávanie a online kurzy sa stali neoddeliteľnou časťou kultúry vzdelávania a šírenia informácií. Rastúci dopyt po tomto type vzdelávania v je veľmi dobre zdokumentovaný, takmer tretina študentov v súčasnosti absolvuje počas stredoškolského alebo vysokoškolského štúdia aspoň 1 online kurz. Viacero štúdií ukázalo, že tento typ vzdelávania môže byť lepší ako tradičné spôsoby. Online výučba prebieha pomocou videí s komentármi lektorov aj s názornými ukážkami, alebo webinárov, v ktorých lektori prednášajú priamo cez videohovor. Z videí je možné sa učiť čokoľvek prakticky od nuly, napríklad nový jazyk, základy hry na gitare, tvorby webstránok a mnoho ďalších vecí. \cite{havrish2019implementation}

\section{Výhody a nevýhody online kurzov} \label{nejaka}

\subsection{Výhody}

\begin{itemize}
  \item \textbf{Časová flexibilita} - je najčastejším dôvodom, prečo ľudia uprednostňujú online vzdelávanie pred prezenčnou formou. Ľudia si môžu sami určiť, kedy sa budú vzdelávať, či už je to skoro ráno, alebo večer po pracovnej dobe. Každému vyhovuje niečo iné, niektorí ľudia si radi urobia pauzu kedy im to vyhovuje, nie kedy im to určuje niekto iný.
  \item \textbf{Miesto} - každý môže študovať, tam kde sa cíti najpohodlnejšie, napríklad v kuchyni, na gauči alebo aj vonku. Môže byť prakticky všade, kde má prístup k internetu, prípadne elektrickej sieti. Najnovším trendom sú takzvané "online kaviarne", ktoré ľudia navštevujú kvôli kľudu, rýchlemu internetovému pripojeniu a prípadnému občerstveniu.
  \item \textbf{Náklady} - pri online kurzoch nevznikajú ďalšie zbytočné náklady na  cestovanie alebo ubytovanie ako pri prezenčných kurzoch, ktoré sa môžu nachádzať ďaleko od bydliska. V takýchto prípadoch je potrebné bookovanie hotela, cestovné lístky a podobné nepríjemné okolnosti.
  \item \textbf{Možnosť výberu} - sú tisíce stránok s online kurzami, či už zahraničné alebo domáce. Každý si vyberie podľa vlastných preferencií dĺžky, jazyka alebo ceny kurzu. Nespočet stránok poskytuje kurzy práve zadarmo.
  \item \textbf{Komunikácia} - online kurzy výrazne prispievajú k interakcii študentov a to tým, že študenti môžu viesť debatu spolu cez email alebo chat, kde môžu zdieľať svoje nápady a myšlienky. Taktiež inštruktori sú prístupnejší v online prostredí. Študenti sa tak cítia pohodlnejšie, uvoľnene a otvorenejšie než pri komunikácii tvárou v tvár. Okrem toho, výhoda je aj to, že nemusíte čakať na konzultačné hodiny vášho učiteľa.
\end{itemize}

\subsection{Nevýhody}

\begin{itemize}
  \item \textbf{Sústredenosť} - pri študovaní v "pohodlí domova", je niekedy veľmi obtiažne udržať pozornosť a sústredenosť na daný kurz. Doma sa nachádza nespočet rušivých faktorov, ktoré môžu výrazne ovplyvniť množstvo informácií odnesených z kurzu. 
  \item \textbf{Kvalita} - rôzne kurzy zadarmo síce naučia veľa nových vecí, no často nestačia na získanie dostatku odborných znalostí do prípadného zamestnania. Závisí od certifikácie a typu kurzu, no zvyčajne je potrebné ďalšie vzdelávanie. Po absolvovaní niektorých kurzov, sú ľudia niekedy sklamaní, z rozsahu prezentovaného učiva, z prejavu inštruktora alebo iných nedostatkov.
  \item \textbf{Motivácia} - pre ľudí, čo nikdy neskúšali samovzdelávanie možno prekvapí, že je veľmi nárožné ostať motivovaný počas celého trvania kurzu. Ľahko sa môže stať, že prestane kurzu venovať pozornosť a radšej sa rozhodne robiť niečo iné, väčšinou zbytočné. V tomto sú prezenčné kurzy lepšie, lebo nútia daného človeka dodržiavať striktné pravidlá, inak sa mu kurz nepodarí dokončiť.
\end{itemize}

\section{Dôležitá časť} \label{dolezita}

\section{Záver} \label{zaver}

\bibliography{literatura}
\bibliographystyle{plain}
\end{document}
